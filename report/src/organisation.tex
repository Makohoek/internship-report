\chapter{Organisation}

\section{Planning}
% What i did, when, how long
\subsection{Gantt}
% Disovering the team, the PWF
% Working with them within the scrum: xml stories
%

\section{Agile methodology}

In the team I was working, we were organized around Agile software
development. This way of working promotes incremental results and adaptive planning.
There are several implementations of Agile methods. In our team we were using Scrum.

In order to understand how I worked during my internship, some key concepts of Scrum must be described.

\subsection{Core concepts}

\subsubsection{Story}\label{sec:story}
A story is a \emph{business-oriented}, short description of a client's
need. Most of the times, it is divided into several tasks so that the developers
can take small steps to complete the story. A story is usually printed or
written on sticky note. Those notes are grouped on a story board.

\begin{figureGraphics}{A story board}{fig:storyboard}
    TODO picture of a story board
\end{figureGraphics}


\subsubsection{Sprints}\label{sec:sprint}
Sprints are short development cycles, usually from one to four weeks. We were
delivering results every three weeks. During each sprint, a team creates a
shippable product, no matter how basic that product is. A sprint is composed of
a set of stories which should be completed at it's end.

The interesting part of working in sprints is the idea of "making a fresh start" every three weeks. This
was quite motivating for me!


\subsection{Team}
Our team is composed of six members(me included); all software engineers with various
skills. Some are more software design oriented while others are more hardware
specialists.

In a Scrum, some members have a particular role:
\begin{description}
    \item[The Product owner]
        defines what the team is doing during a Sprint (see \ref{sec:sprint}).
        He determines the priority of each story (see \ref{sec:story}) the team is working on. His input usually comes from
        clients. He is the business-oriented person in the team. His decisions have an impact on the results a Scrum team delivers.
    \item[The Scrum master]
        ensures that every team member is correctly focused on his story. He is keeping track of the progress of each member and
        should alert the Product owner when some planning issues occur (bad time estimation, very urgent incoming task, ...).
\end{description}

Note that in our team, the Product owner and the Scrum master were also contributing to the team as software engineers.

\subsection{Events}
In Scrum, there are several events occurring during TODO

\subsubsection{Daily scrum}
Every day, at 11:30 we were holding the daily scrum meeting, also called "stand-up".
During that time, each member of the team answers quickly the three following questions:

\begin{description}
    \item[What have I done since yesterday ?]
    \item[What am I planning for today ?]
    \item[Any issues encountered ?]
\end{description}

This meeting is very useful, it helps tackle early problems and can assist the scrum master to detect delays in delivering.
Note that this should not take longer than 15 minutes.


\subsubsection{Poker planning}

\subsubsection{Demos}
\subsubsection{Retrospectives}

\section{Workflow}
\subsection{Setup}
\subsection{Development}
\subsection{Verification}
\subsection{Review}
\subsection{Pre-integration}
