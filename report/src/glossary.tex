% Thanks to Apelete Seketeli <apelete@seketeli.org> for this glossary.

% $0 glossary entries: UMG, MeeGo, Android, PSI, Medfield, MID, SoC

% define an acronym using \newglossaryentry
%\newglossaryentry{umg}
%{
% how the entry name should appear in the glossary
%  name=UMG (Ultra Mobility Group),
%% how the description should appear in the glossary
%  description={Intel's department responsible for focusing core
%    processor and chipset business on mobile platforms},
%% how the entry should appear in the document text
%  text=UMG,
%% how the entry should appear the first time it is used in the
%% document text
%
%  first=Ultra Mobility Group (UMG)
%}

\newglossaryentry{umg}
{
  name=UMG (Ultra Mobility Group),
  description={Intel's department responsible for focusing core
    processor and chipset business on mobile platforms},
  text=UMG,
  first=Ultra Mobility Group (UMG)
}

\newglossaryentry{meego}
{
  name=MeeGo\textsuperscript{\texttrademark},
  description={A Linux-based open source mobile operating system
    jointly created by Intel and Nokia}
}

\newglossaryentry{android}
{
  name=Android\textsuperscript{\texttrademark},
  description={Google's software stack for mobile devices, based on
  the Linux kernel, that includes an operating system, middleware and
  key applications}
}

\newglossaryentry{hal}
{
  name=HAL,
  description={A set of routines that eases the access to a
  platform's hardware resources.},
  first=Hardware Abstraction Layer (HAL)
}

\newglossaryentry{gerrit}
{
  name=Gerrit,
  description={Android's official web based code review system.
  It uses Git as version control system}
}

\newglossaryentry{pfw}
{
  name=Parameter-framework,
  description={Intel Audio team's generic solution to allow system-wide
  parameter management. It is configured via Structure and Settings files.
  It relies on an open plugin architecture to support systems such as Audio,
  Bluetooth, Energy Management, etc...
  }
}

\newglossaryentry{scrum}
{
  name=Scrum,
  description={One of the most known implementation of Agile software development.
  It is a flexible, holistic product development strategy where a development team
  works as a unit to reach a common goal}
}

\newglossaryentry{markdown}
{
  name=Markdown,
  description={A plain-text based language which is used to quickly write documents.
  Those files can be converted into HTML, LaTex, PDF, or other formats.
  It is extremely popular for readme files.}
}

\newglossaryentry{pullrequests}
{
  name=pull requests,
  description={A standard way to contribute to git projects. Based on the fork \& pull model.
  See \url{https://help.github.com/articles/using-pull-requests} for more information}
}

\newglossaryentry{python}
{
  name=Python,
  description={A widely used multi-purpose language, with a huge standard library.
  It is mostly used for scripting tasks, but can be used for everything.
  Python was invented by Guido van Rossum, in Holland}
}

\newglossaryentry{aosp}
{
  name=AOSP,
  description={The Android open-source project led by Google},
  first=Android OpenSource Project (AOSP)
}

\newglossaryentry{alsa}
{
  name=ALSA,
  description={Linux kernel component responsible of handling the Linux audio stack},
  first=Advanced Linux Sound Architecture (ALSA)
}

\newglossaryentry{psi}
{
  name=PSI (Platform Software Integration),
  description={A division of UMG, responsible for the integration and
    validation of the software components for the future Intel mobile
    phone platform},
  text=PSI,
  first=Platform Software Integration (PSI)
}

\newglossaryentry{medfield}
{
  name=Medfield,
  description={Intel's fourth generation Mobile Internet Device
    platform. It is build around an Intel System-on-Chip which
    contains a 32nm Intel Atom\textsuperscript{\texttrademark}
    processor}
}

\newglossaryentry{mid}
{
  name=MID (Mobile Internet Device),
  description={A multimedia-capable mobile device providing wireless
    Internet access},
  text=MID,
  first=Mobile Internet Devices (MID)
}

\newglossaryentry{soc}
{
  name=SoC (System-on-Chip),
  description={Refers to integrating all components of a computer or
    other electronic system into a single integrated circuit},
  text=SoC,
  first=System-on-Chip (SoC)
}

% $1 glossary entries: Atom, ODM, OS, Autotools, Makefile, Logic
% analyzer, GDB, strace, Git, Kernel, Latex, Penwell, MSIC, 3G, iCDK,
% eMMC, kboot, Firmware

\newglossaryentry{atom}
{
  name=Intel Atom\textsuperscript{\texttrademark},
  description={The brand name for a line of ultra-low-voltage x86 and
    x86-64 microprocessors from Intel, used mainly in netbooks and
    Mobile Internet Devices},
  text=Atom,
  first=Intel Atom\textsuperscript{\texttrademark}
}
\newglossaryentry{xml}
{
  name=XML (Extensible Markup Language),
  description={A set of rules for encoding documents in
    machine-readable form. The design goals of XML emphasize
    simplicity, generality, and usability over the Internet. It is a
    textual data format with strong support via Unicode for the
    languages of the world. Although the design of XML focuses on
    documents, it is widely used for the representation of arbitrary
    data structures, for example in web services},
  text=XML,
  first=Extensible Markup Language (XML)
}

\newglossaryentry{xsd}
{
  name=XSD,
  description={
    XSD can be used to express a set of rules to which an XML document must
    conform in order to be considered 'valid' according to that schema
  },
  first=XML Schema Definition (XSD)
}

\newglossaryentry{odm}
{
  name=ODM (Original Design Manufacturer),
  description={A company which designs and manufactures a product
    which is specified and eventually branded by another firm for
    sale},
  text=ODM,
  first=Original Design Manufacturers (ODM)
}

\newglossaryentry{msic}
{
  name=MSIC (Mixed Signal Integrated Circuit),
  description={Any integrated circuit that has both analog circuits
    and digital circuits on a single semiconductor die},
  text=MSIC,
  first=Mixed Signal Integrated Circuit (MSIC)
}

\newglossaryentry{os}
{
  name=OS (Operating System),
  description={A software that runs on computers, manages computer
    hardware resources, and provides common services for execution of
    various application software. Without an operating system, a user
    cannot run an application program on his computer, unless the
    application program is self booting},
  text=OS,
  first=Operating Systems (OS)
}

\newglossaryentry{autotools}
{
  name=Autotools,
  description={The GNU build system, also known as the Autotools, is a
    suite of programming tools designed to assist in making
    source-code packages portable to many Unix-like systems},
  text=autotools
}

\newglossaryentry{makefile}
{
  name=Makefile,
  description={In software development, ``make'' is a utility that
    automatically builds executable programs and libraries from source
    code, by reading files called makefiles which specify how to
    derive the target program},
  text=makefile
}

\newglossaryentry{logic analyzer}
{
  name=Logic Analyzer,
  description={An electronic instrument which displays signals in a
    digital circuit. For PC-based analyzers, the hardware connects to
    a computer through a USB or Ethernet connection and then relays
    the captured signals to the software on the computer},
  text=logic analyzer
}

\newglossaryentry{gdb}
{
  name=GDB (GNU Debugger),
  description={The GNU Debugger is the standard debugger for the GNU
    software system. It is a portable debugger that runs on many
    Unix-like systems and works for many programming languages},
  text=GDB,
  first=GNU Debugger (GDB)
}

\newglossaryentry{strace}
{
  name=Strace,
  description={A debugging utility in Linux to monitor the system
    calls used by a program and all the signals it receives},
  text=strace
}

\newglossaryentry{git}
{
  name=Git,
  description={A distributed revision control system with an emphasis
    on speed. Git was initially designed and developed by Linus
    Torvalds for Linux kernel development},
}

\newglossaryentry{vim}
{
  name=Vim,
  description={A modal text editor, based on vi, and maintained by Bram Moolenaar.
  It power comes from it's different modes, motions, and it's extensibility.}
}

\newglossaryentry{tmux}
{
  name=Tmux,
  description={The terminal multiplexer. Is very useful to organize multiple terminal instances in
  one terminal emulator. It handles splits, multiple pane layouts, and is fully configurable via a
  configuration file}
}

\newglossaryentry{kernel}
{
  name=Linux kernel,
  description={An operating system kernel used by the Linux family of
    Unix-like operating systems. It is one of the most prominent
    examples of free and open source software},
  text=kernel
}

\newglossaryentry{latex}
{
  name=\LaTeX,
  description={A document markup language and document preparation
    system for the TeX typesetting program. The term LaTeX refers only
    to the language in which documents are written, not to the editor
    used to write those documents},
}

\newglossaryentry{penwell}
{
  name=Penwell,
  description={Le nom de code du system-on-Chip qui contient le notament le processeur
   central Atom},
  text=Penwell
}

\newglossaryentry{ifx}
{
  name=IFX (Infineon),
  description={The codename of the Infineon modem on the software
    development plateform. Infineon Technologies AG is a German
    semiconductor manufacturer and was founded in 1999, when the
    semiconductor operations of the parent company Siemens AG were
    spun off to form a separate legal entity},
  text=IFX,
  first=Infineon (IFX)
}

\newglossaryentry{3g}
{
  name=3G,
  description={Also known as ``3rd generation mobile
    telecommunications'', it is a generation of standards for mobile
    phones and mobile telecommunication services. Application services
    include wide-area wireless voice telephone, mobile Internet
    access, video calls and mobile TV, all in a mobile
    environment. Recent 3G releases, often denoted 3.5G and 3.75G,
    also provide mobile broadband access of several Mbit/s to
    smartphones and mobile modems in laptop computers}
}

\newglossaryentry{ti}
{
  name=TI (Texas Instrument),
  description={An American company founded in 1930, which develops and
    commercializes semiconductor and computer technology. Texas
    Instrument is one of the largest manufacturer of semiconductors
    worldwide after Intel},
  text=TI,
  first=Texas Instrument (TI)
}

\newglossaryentry{icdk}
{
  name=iCDK (Intel Customer Development Kit),
  description={The codename of the Intel software development
    platform. It is a printed circuit board containing an Intel
    System-on-Chip, electronic components and peripherals from various
    hardware manufacturers},
  text=iCDK,
  first=Intel Customer Development Kit (iCDK)
}

\newglossaryentry{emmc}
{
  name=eMMC (Embedded MultiMedia Card),
  description={A flash memory card standard. eMMC describes an
    architecture consisting of an embedded storage solution, flash
    memory and controller, all in a small package},
  text=eMMC
}

\newglossaryentry{kboot}
{
  name=Kboot,
  description={A proof-of-concept implementation of a Linux
    bootloader. This relatively small program is needed to access the
    nonvolatile devices from which the operating system programs and
    data are loaded},
  text=kboot
}

\newglossaryentry{firmware}
{
  name=Firmware,
  description={A fixed, usually rather small, program and/or data
    structure that internally control various electronic devices such
    as remote controls, calculators, hard disks, keyboards or memory
    cards},
  text=firmware
}

% $2 glossary entries: Pulseaudio, PCM, ACS, API, Driver, SSP, I2S,
% KTS

\newglossaryentry{pulseaudio}
{
  name=PulseAudio,
  description={A cross-platform, networked sound server commonly used
    on the Linux-based and FreeBSD operating systems. One of the goals
    of PulseAudio is to reroute all sound streams through it,
    including those from processes that attempt to directly access the
    hardware}
}

\newglossaryentry{pcm}
{
  name=PCM (Pulse-Code Modulation),
  description={A method used to digitally represent sampled analog
    signals. It is the standard form for digital audio in computers as
    well as digital telephone systems},
  text=PCM,
  first=Pulse-Code Modulation (PCM)
}

\newglossaryentry{acs}
{
  name=ACS (Automation Control System),
  description={An automation and test sequencing tool developed by
    Intel. It is written in Python and serves test requirements at the
    middleware level. It was chosen as a solution to automate the
    audio quality checker tool},
  text=ACS,
  first=Automation Control System (ACS)
}

\newglossaryentry{api}
{
  name=API (Application Programming Interface),
  description={A particular set of rules (`code') and specifications
    that software programs can follow to communicate with each
    other. It serves as an interface between different software
    programs and facilitates their interaction},
  text=API,
  first=Application Programming Interface (API)
}

\newglossaryentry{driver}
{
  name=Device driver,
  description={A computer program allowing higher-level computer
    programs to interact with a hardware device},
  text=driver,
  first=device drivers
}

\newglossaryentry{ssp}
{
  name=SSP (Synchronous Serial Port),
  description={A controller that supports the Serial Peripheral
    Interface and Integrated Interchip Sound interface. It uses a
    master-slave paradigm to communicate across its connected bus},
  text=SSP,
  first=Synchronous Serial Port (SSP)
}


\newglossaryentry{kts}
{
  name=KTS (Kernel Test Suite),
  description={A suite of test programs for audio quality assurance at
    the driver level. The programs in this suite are aimed at the
    audio hardware devices in Medfield, but currently provide test
    routines for the bluetooth and modem chip},
  text=KTS,
  first=Kernel Test Suite (KTS)
}

% $3 glossary entries: LPE, DMAC, FIFO, MISO, MOSI, SPI, BlueZ, HCI,
% UART, IOCTL, ssploop, sspconf, AT commands, KGDB, dmesg, GPL, Free
% Software, Gitorious, XML

\newglossaryentry{lpe}
{
  name=LPE (Low Power Engine),
  description={A circuit of the Intel System-on-Chip responsible for
    audio processing. It is designed to process Pulse-Code Modulation
    data as well as compressed audio data while being power efficient},
  text=LPE,
  first=Low Power Engine (LPE)
}

\newglossaryentry{dmac}
{
  name=DMAC (Direct Memory Access Controller),
  description={A controller that allows certain hardware subsystems
    within the computer to access system memory for reading and/or
    writing independently of the central processing unit. Without
    Direct Memory Access, the central processing unit is typically
    fully occupied for the entire duration of the read or write
    operation, and is thus unavailable to perform other work. With
    Direct Memory Access, the central processing unit would initiate
    the transfer, do other operations while the transfer is in
    progress, and receive an interrupt from the Direct Memory Access
    Controller once the operation has been done},
  text=DMAC,
  first=Direct Memory Access Controller (DMAC)
}

\newglossaryentry{fifo}
{
  name={FIFO (First In, First Out)},
  description={An abstraction in ways of organizing and manipulation
    of data relative to time and prioritization. This expression
    describes the principle of a queue processing technique: what
    comes in first is handled first, what comes in next waits until
    the first is finished},
  text=FIFO,
  first={First In, First Out (FIFO)}
}

\newglossaryentry{miso}
{
  name={MISO (Master Input, Slave Output)},
  description={A data line in the Serial Peripheral Interface and
    Integrated Interchip Sound buses. The name means that data on this
    line is outputed from the slave device},
  text=MISO,
  first={Master Input, Slave Output (MISO)}
}

\newglossaryentry{mosi}
{
  name={MOSI (Master Output, Slave Input)},
  description={A data line in the Serial Peripheral Interface and
    Integrated Interchip Sound buses. The name means that data on this
    line is outputed from the master device},
  text=MOSI,
  first={Master Output, Slave Input (MOSI)}
}

\newglossaryentry{spi}
{
  name=SPI (Serial Peripheral Interface),
  description={A synchronous serial data link standard named by
    Motorola that operates in full duplex mode},
  text=SPI,
  first=Serial Peripheral Interface (SPI)
}

\newglossaryentry{bluez}
{
  name=BlueZ,
  description={The canonical Bluetooth stack for Linux. Its goal is to
    make an implementation of the Bluetooth wireless standards
    specifications for Linux. As of 2006, the BlueZ stack supports all
    core Bluetooth protocols and layers. It was initially developed by
    Qualcomm, and is available for Linux kernel versions 2.4.6 and up}
}

\newglossaryentry{hci}
{
  name=HCI (Host/Controller Interface),
  description={A protocol of the Bluetooth stack. It standardizes
    communication between the host stack (e.g., a personal computer or
    mobile phone Operating System) and the controller (the Bluetooth
    integrated circuit). There are several Host/Controller Interface
    transport layer standards, each using a different hardware
    interface to transfer the same command, event and data
    packets. The most commonly used are USB (in personal computers)
    and UART (in mobile phones)},
  text=HCI,
  first=Host/Controller Interface (HCI)
}

\newglossaryentry{uart}
{
  name=UART (Universal Asynchronous Receiver/Transmitter),
  description={An individual (or part of an) integrated circuit used
    for serial communications over a computer or peripheral device
    serial port. It is now commonly included in micro-controllers},
  text=UART,
  first=Universal Asynchronous Receiver/Transmitter (UART)
}

\newglossaryentry{ioctl}
{
  name=IOCTL (Input/Output Control),
  description={A system call for device-specific operations and other
    operations which cannot be expressed by regular system calls. It
    takes a parameter specifying a request code; the effect of a call
    depends completely on the request code},
  text=IOCTL,
  first=Input/Output Control (IOCTL)
}

\newglossaryentry{ssploop}
{
  name=Ssploop,
  description={A loopback test program which is part of the utilities
    found in the Kernel Test Suite. In its current state it provides
    test routines for the Infineon modem and the Texas Instrument
    bluetooth chips},
  text=ssploop
}

\newglossaryentry{sspconf}
{
  name=Sspconf,
  description={A configuration program which is part of the utilities
    found in the Kernel Test Suite. In its current state it provides
    configuration capabilities for the Synchronous Serial Port
    controller on the Intel System-on-Chip associated to the Texas
    Instrument bluetooth chip},
  text=sspconf
}

\newglossaryentry{at commands}
{
  name=AT Commands,
  description={Also known as ``Hayes command set'', it is a specific
    command-language originally developed for the Hayes Smartmodem 300
    baud modem in 1981 by Hayes Communications. The command set
    consists of a series of short text strings which combine together
    to produce complete commands for operations such as dialling,
    hanging up, and changing the parameters of the connection}
}

\newglossaryentry{kgdb}
{
  name=KGDB,
  description={A debugger for the Linux kernel. It requires two
    machines that are connected via a serial connection. It was
    originally implemented as a patch to Linux kernel, but it has been
    included in the official kernel in 2.6.26. The target machine (the
    one being debugged) runs the patched kernel and the other (host)
    machine runs the GNU Debugger. The GNU Debugger remote protocol is
    used between the two machines},
}

\newglossaryentry{dmesg}
{
  name=Dmesg,
  description={A command on most Linux and Unix based operating
    systems that prints the message buffer of the kernel},
  text=dmesg
}

\newglossaryentry{gpl}
{
  name=GPL (GNU General Public License),
  description={The most widely used Free Software license, originally
    written by Richard Stallman for the GNU project. It is the first
    copyleft license for general use, which means that derived works
    can only be distributed under the same license terms. Under this
    philosophy, the GNU Public License grants the recipients of a
    computer program the rights of the free software definition and
    uses copyleft to ensure the freedoms are preserved, even when the
    work is changed or added to},
  text=GPL,
  first=GNU General Public License (GPL)
}

\newglossaryentry{free software}
{
  name=Free Software,
  description={Software that can be used, studied, and modified
    without restriction, and which can be copied and redistributed in
    modified or unmodified form either without restriction, or with
    restrictions that only ensure that further recipients can also do
    these things and that manufacturers of consumer-facing hardware
    allow user modifications to their hardware. In practice, for
    software to be distributed as free software, the human-readable
    form of the program (the source code) must be made available to
    the recipient along with a notice granting the above
    permissions. Such a notice either is a free software license, or a
    notice that the source code is released into the public domain},
}

\newglossaryentry{gitorious}
{
  name=Gitorious,
  description={A website hosting collaborative open source projects
    using the Git distributed revision control system. It is used
    internally by Platform Software Intergration engineering teams to
    manage code delivery},
}



% $4 glossary entries: CAMSI

\newglossaryentry{camsi}
{
  name=CAMSI,
  description={Stands for ``Concepteur en Architectures de Machines et
    Systèmes Informatiques''. It is a Master Degree diploma
    issued by Université Paul Sabatier in Toulouse. The courses of this
    diploma include electronic layout design, computer architecture design,
    micro-controller programming, Linux kernel programming and
    real-time Operating Systems scheduling}
}

% easter-egg glossary entry: Toulouse

\newglossaryentry{toulouse}
{
  name=Toulouse,
  description={A middle-sized city in southwest France with friendly
    people and an almost constantly sunny weather, albeit windy}
}

