\chapter{Achievements}

\begin{sectionIntro}
    All the work I have done during this internship is related
    to the Parameter-framework, which was presented in section \ref{sec:parameter-framework}.
\end{sectionIntro}

% Context will be moved to another place
%\section{Context}
%
%\subsection{Android architecture}
%
%\subsection{Audio XML HAL}
%Scalable, fully configurable, userland
%
%\subsection{Parameter framework}
%Middleware, gap, no standard

\section{Defining the new middleware standard}
\subsection{Discovery tutorials}

At the start of my internship, I had to check out the Parameter-framework's
source code without any documentation. The idea was to have a fresh look at
this piece of software to determine if it was open-source ready, straightforward
to use, for someone who is unfamiliar with it.

While doing that, I struggled a bit with the basic usage of the framework. The
team decided that it would be nice to have some newcomer tutorials and examples,
for an easier early-adoption of the open-source community. So I wrote several
tutorials:
\begin{description}
    \item[Compile and install]
        is a step-by-step guide about how to get the Parameter-framework's sources,
        build it and install it as a standalone on Ubuntu.
    \item[Run a simple example]
        is a howto about running the Parameter-framework command-line interface, such
        as \lstinline{remote-process} and \lstinline {test-platform}.
        In this howto, the configuration and setting files are provided so that
        the user can focus on the results.
    \item[An introduction to the .pfw language]\label{desc:pfw-language}
        is a tutorial about the .pfw language. This language was
        created to simplify the writing of settings files for the
        Parameter-framework. Those files are then converted into XML, which is
        the only language the Parameter-framework understands.
\end{description}

\subsection{Open-sourcing on GitHub}
Pull requests to simulate external contributions.

\section{Improving build process}
\subsection{Build process}
The pfw language as described here\ref{desc:pfw-language} is used for rule based
description. Need structure files as well, which are written in XML.

\begin{figureGraphics}{Xml generation build process}{ref:build-proces}
    TODO
\end{figureGraphics}

Some constraints may apply. Writting XML is error prone, and no check was done at
build time. Lead to run-time errors and undefined behaviour.
\subsection{Xml checker}
\subsection{Schemas}

\section{Multi-variant initial support}
\subsection{Alsa}
\subsection{parameter-framework}

\section{Remote parameter enhancements}

\section{Fixed point parameter enhancements}
The parameter-framework has several kinds of parameters:
\begin{itemize}
    \item IntegerParameter
    \item BooleanParameter
    \item FixedPointParameter
    \item StringParameter
    \item EnumParameter
    \item And more..
\end{itemize}

Fixed point numbers are useful for representing fractional numbers. Their usage
is appropriate when the processor does not haves a floating point unit or when
there is a performance gain by using them.

The implementation of fixed point numbers in the Parameter-framework had some
corner cases which were not correctly handled. Round issues appeared when we
were writing to a configuration file or displaying user input. In order to
support the expected behaviour for those cases, I changed the implementation, and
wrote tests for it.

\subsection{Test suite in python}
\subsection{Rework display mechanism}

\section{Parameter-framework's intellectual property}
BSD
Private
\subsection{license checker}
Python internal tool

\section{Alsa plugin refactoring}
Not the same controls, some proprietary code was mixed with open-source code.
Some part of the code is Intel specific. We do not wish to open-source that.

\subsection{Alsa}
\subsection{Tinyalsa}
\subsection{Plugin architecture}
CTL and MIX merged

\section{Debug and trace participation}

\section{File system plugin upload}

\section{Core upload}
Open-sourcing the Parameter-framework required some initial work.
In order to smoothly integrate features from external contributors and internal work,
Requirements, vanilla AOSP, internal tree and open-source version must be the
same.
01org organisation, Intel opensource technology center.

\section{Multiple modems handling(DSDA)}
TODO

\section{Porting the Intel Audio HAL on a new platform}
TODO
\subsection{Pandaboard}

