\chapter{Achievements}

\begin{sectionIntro}
    All the work I have done during this internship is related
    to the Parameter-framework, which was presented in section \ref{sec:parameter-framework}.
\end{sectionIntro}

\section{Parameter-framework introduction}
\subsection{Discovery tutorials}

At the start of my internship, I had to check out the Parameter-framework's
source code without any documentation. The idea was to have a fresh look at
this piece of software to determine if it was open-source ready, straightforward
to use, for someone who is unfamiliar with it.

While doing that, I struggled a bit with the basic usage of the framework. The
team decided that it would be nice to have some newcomer tutorials and examples,
for an easier early-adoption of the open-source community. So I wrote several
tutorials:
\begin{description}
    \item[Compile and install]
        is a step-by-step guide about how to get the Parameter-framework's sources,
        build it and install it as a standalone on Ubuntu.
    \item[Run a simple example]
        is a howto about running the Parameter-framework command-line interface,
        such as \lstinline{remote-process} and \lstinline {test-platform}.  In
        this howto, the configuration and setting files are provided so that
        the user can focus on the results. The example covers music play-list
        changing based on a user's mood.
    \item[An introduction to the .pfw language]\label{desc:pfw-language}
        is a tutorial about the .pfw language. This language was
        created to simplify the writing of settings files for the
        Parameter-framework. Those files are then converted into XML, which is
        the only language the Parameter-framework understands.
\end{description}

\subsection{Open-sourcing on GitHub}
Pull requests to simulate external contributions.
set up for easy internal tree contributions and external contributions.

\begin{figureGraphics}{Branching process}{ref:branch-process}
    Figure: branching process
\end{figureGraphics}

\section{Improving build process}
\subsection{Build process}
The pfw language as described in \ref{desc:pfw-language} is used for rule based
description. Need structure files as well, which are written in XML.

\begin{figureGraphics}{Xml generation build process}{ref:build-process}
    TODO
\end{figureGraphics}

Some constraints may apply. Writting XML is error prone, and no check was done at
build time. Lead to run-time errors and undefined behaviour.
\subsection{Xml checker}
\subsection{Schemas}

\section{Multi-variant initial support}
\subsection{Alsa}
\subsection{parameter-framework}

\section{Remote parameter enhancements}

\section{Fixed point parameter enhancements}
The parameter-framework has several kinds of parameters:
\begin{itemize}
    \item IntegerParameter
    \item BooleanParameter
    \item FixedPointParameter
    \item StringParameter
    \item EnumParameter
    \item ...
\end{itemize}

Fixed point numbers are useful for representing fractional numbers. Their
usage is appropriate when the processor does not haves a floating point unit
or when there is a performance gain by using them. Usually, fixed points are
represented by $Qn.m$ format, where $n$ is the \emph{Integral part} and $m$ the
\emph{Fractional part}.

The implementation of fixed point numbers in the Parameter-framework had some
corner cases which were not correctly handled. Round issues appeared when we
were writing to a configuration file or displaying user input. To
support the expected behaviour for those cases, I changed the implementation, and
wrote tests for it.

\subsection{Test suite in python}

In order to verify that the new implementation was correct, I wrote a test suite
in Python. That test suite performs several checks on each fixed point. Let's
take an example to illustrate the purpose of the test suite.

Let's take the case of a $Q2.3$ number: the \emph{Integral part} is $2$ and the
\emph{Fractional part} is $3$. We can compute some specific values for that number:

\begin{figureGraphics}{$Q.2.3$ special values}{fig:fixedPoint}
    TODO special values for a Q.2.3 number
\end{figureGraphics}

The test suite takes each of these special values and performs four checks on
them:

\begin{figureGraphics}{Fixed point test suite}{fig:fixedPointTest}
    TODO figure interaction between PFW and python test script.
\end{figureGraphics}

\begin{description}
    \item[Bound check] The Parameter-framework should throw an error if we
        attempt to set an out-of-bound value.
    \item[Sanity check] If we manage to set the value, The Parameter-framework should not modify too much
        the value. It can only change by a quantum.
    \item[Consistency check] The Parameter-framework should accept the value he sent us previously.
    \item[Bijectivity check] The Parameter-framework should return us the same value we provided him at the Consistency check.
\end{description}

After that the test suite was complete, it was time to rework the internal
mechanism of the Parameter-framework.

\subsection{Rework the internal mechanism}
The Parameter-framework can export parameters towards a file. Fixed
point parameters can be exported as well.
\begin{itemize}
    \item When exporting them, the Parameter-framework converts the value from
        its internal representation towards a floating point number, because that is
        easier to read.
    \item By converting that number, it also computes the amount of digits
        to use for display, or writing towards a file. This can result in
        \emph{rounding issues}, due to old C++ limitations.
\end{itemize}
The fix I proposed was to replace the computation of displayable digits by something
easier, which is the \emph{Fractional part} of the fixed point number.

The example below illustrates the rounding error, and the output in the corrected version.
\begin{figureGraphics}{$Q.2.3$ rounding issue}{fig:fixedPointProblem}
    TODO show limitation example here and the new solution\\
\end{figureGraphics}


\section{Parameter-framework's intellectual property}
BSD
Private
\subsection{license checker}
Python internal tool

\section{Alsa plugin refactoring}
Not the same controls, some proprietary code was mixed with open-source code.
Some part of the code is Intel specific. We do not wish to open-source that.

\subsection{Alsa}
\subsection{Tinyalsa}
\subsection{Plugin architecture}
CTL and MIX merged

\section{Debug and trace participation}

\section{File system plugin upload}

\section{Core upload}
Open-sourcing the Parameter-framework required some initial work.
In order to smoothly integrate features from external contributors and internal work,
Requirements, vanilla AOSP, internal tree and open-source version must be the
same.
01org organisation, Intel opensource technology center.

\section{Multiple modem support (DSDA)}
\subsection{Parameter-framework plugin}

\section{Porting the Intel Audio HAL on a new platform}
TODO
\subsection{Pandaboard}

