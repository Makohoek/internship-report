\chapter{Contributions}\label{chap:contributions}

\begin{sectionIntro}
    All the work I have done during this internship is related
    to the \gls{pfw}, which was briefly presented in section \ref{sec:parameter-framework}.
\end{sectionIntro}

\section{Parameter-framework mechanisms}
Should i really add this?

\section{Parameter-framework core enhancements}

\subsection{Improving build process}
\subsubsection{Build process}
The pfw language as described in \ref{desc:pfw-language} is used for rule based
description. Those are translated into \gls{xml} files during the build via a
specific make target.  In order to generate those \gls{xml} files, we also need
the information of the Structure files, which are written by the integration
engineers.

To generate those files, we rely on a toolset we call \emph{XmlGenerator}.
The figure \ref{fig:build-process} shows how it works:

\begin{figureGraphics}{Xml generation build process}{fig:build-process}
    \includegraphics[height=0.4\textheight]{./src/img/build-generation.pdf}
\end{figureGraphics}

But there were some limitations. Since the writing of the \gls{xml} structure files is done by human beings, there was
\emph{no guarantee} that their files were semantically correct.
Since no check was done during the build-process, the generator could write erronous files because it supposes that
the Structure files are \emph{correct}.
This lead to \emph{run-time errors} and \emph{undefined behaviour}.

\subsubsection{Schemas}

\subsubsection{Xml checker}

\begin{figureGraphics}{Xml generation build process with XSD check}{fig:build-process-reworked}
    \includegraphics[height=0.4\textheight]{./src/img/build-generation.pdf}
\end{figureGraphics}

\subsection{Multi-variant initial support}
\subsubsection{Alsa}
\subsubsection{Parameter-framework}

\subsection{Fixed point parameter enhancements}
The \gls{pfw} has several kinds of parameters:
\begin{itemize}
    \item IntegerParameter
    \item BooleanParameter
    \item FixedPointParameter
    \item StringParameter
    \item EnumParameter
    \item ...
\end{itemize}

Fixed point numbers are useful for representing fractional numbers. Their
usage is appropriate when the processor does not haves a floating point unit
or when there is a performance gain by using them. Usually, fixed points are
represented by $Qn.m$ format, where $n$ is the \emph{Integral part} and $m$ the
\emph{Fractional part}.

The implementation of fixed point numbers in the \gls{pfw} had some
corner cases which were not correctly handled. Round issues appeared when we
were writing to a configuration file or displaying user input. To
support the expected behavior for those cases, I changed the implementation, and
wrote tests for it.

\subsubsection{Test suite in python}

In order to verify that the new implementation was correct, I wrote a test suite
in Python. That test suite performs several checks on each fixed point. Let's
take an example to illustrate the purpose of the test suite.

Let's take the case of a $Q2.3$ number: the \emph{Integral part} is $2$ and the
\emph{Fractional part} is $3$. We can compute some specific values for that number:

\begin{figureGraphics}{$Q.2.3$ special values}{fig:fixedPoint}
    TODO special values for a Q.2.3 number
\end{figureGraphics}

The test suite takes each of these special values and performs four checks on
them:

\begin{figureGraphics}{Fixed point test suite}{fig:fixedPointTest}
    TODO figure interaction between PFW and python test script.
\end{figureGraphics}

\begin{description}
    \item[Bound check] The \gls{pfw} should throw an error if we
        attempt to set an out-of-bound value.
    \item[Sanity check] If we manage to set the value, The \gls{pfw} should not modify too much
        the value. It can only change by a quantum.
    \item[Consistency check] The \gls{pfw} should accept the value he sent us previously.
    \item[Bijectivity check] The \gls{pfw} should return us the same value we provided him at the Consistency check.
\end{description}

After that the test suite was complete, it was time to rework the internal
mechanism of the \gls{pfw}.

\subsubsection{Rework the internal mechanism}
The \gls{pfw} can export parameters towards a file. Fixed
point parameters can be exported as well.
\begin{itemize}
    \item When exporting them, the \gls{pfw} converts the value from
        its internal representation towards a floating point number, because that is
        easier to read.
    \item By converting that number, it also computes the amount of digits
        to use for display, or writing towards a file. This can result in
        \emph{rounding issues}, due to old C++ limitations.
\end{itemize}
The fix I proposed was to replace the computation of displayable digits by something
easier, which is the \emph{Fractional part} of the fixed point number.

The example below illustrates the rounding error, and the output in the corrected version.
\begin{figureGraphics}{$Q.2.3$ rounding issue}{fig:fixedPointProblem}
    TODO show limitation example here and the new solution\\
\end{figureGraphics}


\subsection{Multiple modem support}
\subsubsection{Parameter-framework plugin}

\section{Open-sourcing on GitHub}
Pull requests to simulate external contributions.
Set up for easy internal tree contributions and external contributions.

\begin{figureGraphics}{Branching process}{ref:branch-process}
    TODO
    Figure: branching process
\end{figureGraphics}


\subsection{Parameter-framework introduction}
\subsubsection{Discovery tutorials}\label{sec:tutorials}

At the start of my internship, I had to check out the \gls{pfw}'s
source code without any documentation. The idea was to have a fresh look at
this piece of software to determine if it was open-source ready, straightforward
to use, for someone who is unfamiliar with it.

While doing that, I struggled a bit with the basic usage of the framework. The
team decided that it would be nice to have some newcomer tutorials and examples,
for an easier early-adoption of the open-source community. So I wrote several
tutorials:
\begin{description}
    \item[Compile and install]
        is a step-by-step guide about how to get the \gls{pfw}'s sources,
        build it and install it as a standalone on Ubuntu.
    \item[Run a simple example]
        is a howto about running the \gls{pfw} command-line interface,
        such as \lstinline{remote-process} and \lstinline {test-platform}.  In
        this howto, the configuration and setting files are provided so that
        the user can focus on the results. The example covers music play-list
        changing based on a user's mood.
    \item[An introduction to the .pfw language]\label{desc:pfw-language}
        is a tutorial about the .pfw language. This language was
        created to simplify the writing of settings files for the
        \gls{pfw}. Those files are then converted into \gls{xml}, which is
        the only language the \gls{pfw} understands.
\end{description}
These tutorials have been written in \gls{markdown}, the standard format used
on GitHub.


\subsection{Parameter-framework's intellectual property}
Intellectual property is very important at Intel. TODO.
BSD
Private
\subsubsection{license checker}
Python internal tool

\subsection{Core upload}
Open-sourcing the \gls{pfw} required some initial work.
In order to smoothly integrate features from external contributors and internal work,
Requirements, vanilla AOSP, internal tree and open-source version must be the
same.
01org organisation, Intel opensource technology center.


\subsection{File system plugin upload}

\subsection{Alsa plugin upload}
\subsubsection{Alsa}
\subsubsection{Tinyalsa}
\subsubsection{Plugin architecture}
CTL and MIX merged

\subsection{Alsa plugin refactoring}
Not the same controls, some proprietary code was mixed with open-source code.
Some part of the code is Intel specific. We do not wish to open-source that.

\subsection{Porting the Intel Audio HAL on a new platform}
TODO
\subsubsection{Pandaboard}

