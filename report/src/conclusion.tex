\chapter{Conclusion}

\section{Results}

\subsection{Added value of my work}
\subsubsection{Scrum improvements}
During my internship, I came up with some fresh ideas about how the \gls{scrum} process used within the team.
Several improvements have been made, such as better retrospectives and better demo preparation.
Building team spirit is essential in an effective \gls{scrum} team, something I managed to catalyze within the team.
This has convinced me more than all that \gls{scrum} is simple, but very hard. It is more than just a way to organize
things. It is a different mindset to have. And it is a lot of fun!

\subsubsection{Technical side}
All the contributions listed in chapter \ref{chap:contributions} are integrated into Intel's codebase.
This code is used on a daily basis by thousands of developers. There will even be products shipped running the
code I developed within the team.
As for the open-source documentation, it is available on \gls{GitHub}.

\subsection{Areas of improvement}
Even I did a lot of different things related to the \gls{pfw}, there are still plenty things to do.
In order to stimulate the adoption of the \gls{pfw}, it would be great to
port the Intel Audio \gls{hal} to well-known development boards for example a
Pandaboard, a Raspberry Pi or a Beaglebone.


\section{Personal outcome}

I went to this internship with one goal: a good technical challenge. This goal
was fully reached. This is one of the most complex (and also one of the most
interesting) projects I have ever worked on!

\subsection{Consistency with university classes}
Even if I had a solid technical background thanks to the courses of my
\gls{camsi} Master, there were some skills which I lacked.
I missed especially some skills about advanced \gls{cpp} and some "good software practices".

In contrary, there were also a lot of things which helped me a lot during
my internship.  During the \gls{camsi} Master, we made a lot of stimulating
projects. From creating a simple thermal driver to building our own quadcopter
and make it fly, we really covered various topics! This \emph{all-around} set of
skills helped me to quickly adapt to the requirements at Intel.

\subsection{Career perspectives}
During the middle of my internship, I was actively searching for a job. CELAD offered my a position to work
as contractor at Intel, in the same team I worked with for six months.
Since I learned a \emph{lot} during my internship, I can't wait to continue learning follow my career
as an agile software developer!

\section{Conclusion}
The satisfaction of learning new things is a daily feeling I have at
Intel. Android is huge. The knowledge within the team is huge as well. I can't
wait to contribute more for Intel's Audio team!
