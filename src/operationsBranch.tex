\section{Les opérations de branchement}
Par manque de temps nous avons pu implémenter que les BEQ, BNE, BLTZ, BLEZ, BGTZ et BGEZ.
Les branchements consistent à faire une comparaison de 2 registres c'est à dire une soustraction entre ces 2 registres.
Pour le cas ou il n'y a qu'un registre, par exemple pour le BLEZ, on soustrait avec une valeur zéro codée en dur.
Ensuite, on détermine si on branche grâce à des équations basées sur les flags (N, C, V, Z) correspondante au type de branchement.

\subsubsection{Chemin de données}
\begin{figureGraphics}{Chemin de données opérations de branchement}{fig:chemdonnbranch}
  \centering
  \includegraphics[width=1\textwidth]{./src/img/PipelineBranchs.pdf}
\end{figureGraphics}
\subsubsection{Signaux de contrôle}
N’oublions pas que les signaux non mentionnés ont leur valeur par défaut.
\begin{itemize}
  \item etage EX
    \begin{description}
      \item[ALU\_OP = ALU\_SUB] pour forcer une opération de soustraction
      \item[ALU\_SIGNED = 1] l’opération est signée
      \item[ALU\_SRCA = REGS\_QA] l’ALU prends la valeur RS en opérande A
      \item[ALU\_SRCB = REQS\_QB] l’ALU prends la valeur RT en opérande B
      \item[BRANCHEMENT = 1] Déterminer si on calcule une adresse de branchement qu'on passera à MEM
    \end{description}
  \item etage MEM
    \begin{description}
      \item[BRANCHEMENT = 1] savoir si nous avons une instruction de branchement
      \item[BRA\_TYPE = EQ, NEQ, GT, GE, LT, LE] pour déterminer le type de branchement. Ceci sert à savoir quelles équations sont utilisées pour \emph{confirmer} le branchement
    \end{description}
\end{itemize}
\subsubsection{Tests réalisés}
\lstinputlisting[caption=Test BEQ, label=codeBeq]{../benchs/branchBEQ.asm}
\lstinputlisting[caption=Test BGTZ, label=codeBgtz]{../benchs/branchBGTZ.asm}
\lstinputlisting[caption=Test BLEZ, label=codeBlez]{../benchs/branchBLEZ.asm}
