\section{Les opérations arithmétiques et logiques}

\subsection{Les opérations par registre}
\subsubsection{Chemin de données}
\begin{figureGraphics}{Chemin de données opérations par registre}{fig:chemdonnarithlog}
  \centering
  \includegraphics[width=1\textwidth]{./src/img/PipelineArithmetique.pdf}
\end{figureGraphics}
\subsubsection{Signaux de contrôle}
Afin de s’assurer que ce chemin de données soit respecté, nous avons ajouté dans la procédure de contrôle les signaux suivants:
(Tout les signaux qui ne sont pas marqués ici ont leur valeur par défaut spécifiée dans le code)
\begin{itemize}
  \item etage EX
    \begin{description}
      \item[ALU\_SIGNED = 1] l’opération est signée pour ADD et SUB ( pas pour AND XOR OR NOR)
      \item[ALU\_SRCA = REGS\_QA] l’ALU prends la valeur RS en opérande A
      \item[ALU\_SRCB = REQS\_QB] l’alu prends la valeur RT en operande B
      \item[ALU\_OP = ALU\_ADD] ( si on fait un add par exemple )
      \item[REG\_DST = REG\_RD] le registre de destination (là ou on va écrire le résultat) est le registre RD
    \end{description}
  \item etage ER
    \begin{description}
      \item[REGS\_W = 0] on doit écrire un résultat dans le banc de registres
    \end{description}
\end{itemize}
\subsubsection{Tests réalisés}
Voici les codes exemple que nous avons utilisés : 
\lstinputlisting[caption=Test ADDU, label=codeAdd]{../benchs/addu.asm}
\lstinputlisting[caption=Test SUB, label=codeSub]{../benchs/sub.asm}
\lstinputlisting[caption=Test AND, label=codeAnd]{../benchs/and.asm}

\subsection{Les opérations de décalage}
\subsubsection{Chemin de données}
\begin{figureGraphics}{Chemin de données opérations de décalage}{fig:chemdonnlsl}
  \centering
  \includegraphics[width=1\textwidth]{./src/img/PipelineLSL.pdf}
\end{figureGraphics}
\subsubsection{Signaux de contrôle}
Afin de s’assurer que ce chemin de données soit respecté, nous avons ajouté dans la procédure de contrôle les signaux suivants:
(Tout les signaux qui ne sont pas marqués ici ont leur valeur par défaut spécifiée dans le code)

\begin{itemize}
  \item etage EX
    \begin{description}
      \item[ALU\_SRCB = VAL\_DEC] l’ALU prends la valeur du décalage en opérande B
      \item[ALU\_SRCA = REGS\_QB] l'ALU prends en opérande à la valeur à décaler
      \item[ALU\_OP = ALU\_LSR ou ALU\_LSL] en fonction du décalage
      \item[REG\_DST = REG\_RD] le registre de destination (là ou on va écrire le résultat) est le registre RD ( car opération de format R)
    \end{description}
  \item etage ER
    \begin{description}
      \item[REGS\_W = 0] on doit écrire un résultat du décalage dans le banc de registres
    \end{description}
\end{itemize}
\subsubsection{Tests réalisés}
\lstinputlisting[caption=Test LSL, label=codeLsl]{../benchs/lsl.asm}
