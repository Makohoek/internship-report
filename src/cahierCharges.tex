\chapter{Cahier des charges}

Le processeur à concevoir sera un processeur de type RISC 32 bits et aura les caractéristiques suivantes :
\begin{itemize}
  \item Un banc de registres de 32 registres R0-R31 de 32 bits chacun avec 2 ports de lecture et port d'écriture. Le registre R0 est câblée à 0, on pourra tenter d'y écrire mais sa lecture donnera toujours 0.

  \item  Une unité arithmétique et logique (ALU) avec deux entrées et une sortie de 32 bits. Elle permet de réaliser les opérations arithmétiques et logiques sur 32 bits du jeu d'instructions donnée dans le Tableau 2

  \item Une logique d'états composée de 4 bits : C(Carry), Z(Zero), V(oVerflow), N(Negative) permettant de donner une indication sur l'état des opérations réalisées et de pouvoir effectuer les branchements conditionnels.

  \item Un chemin de données pipeliné de 5 étages à : EI/DI/EX/MEM/ER
    \begin{itemize}
      \item Le premier étage EI réalise l'extraction(lecture) de l'instruction.
      \item Le deuxième étage DI réalise le décodage de l'instruction et extraction des opérandes.
      \item Le troisième étage EX réalise l'exécution et le calcul de l'adresse effective. 
      \item Le quatrième étage MEM réalise l'accès à la mémoire.
      \item Le cinquième étage ER réalise l'écriture du résultat.
    \end{itemize}

  \item Un registre : compteur de programme CP. 

  \item Les modes d'adressages sont de trois types: Registres à Registre, Immédiat, Indirect avec déplacement. Toutes les instructions s'exécutent en un seul cycle. Les opérations d'accès mémoires peuvent être réalisées au niveau de l'octet, le demi-mot (16 bits) et le mot (32 bits) donc les adresses manipulées sont des adresses octets. Toutes les opérations, arithmétiques et logiques sont effectuées toujours sur des opérandes de 32 bits.
\end{itemize}

L'architecture interne du microprocesseur est de type Harvard avec un bus d'adresse et un bus de données pour accéder à une mémoire d'instruction ou de données.

Dans toute l'étude on ne considéra pas l'implémentation des exceptions. 

